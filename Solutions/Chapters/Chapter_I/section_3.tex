\section{Categories}
\extitle

\begin{exercise}
	\(\triangleright\) Let \serif{C} be a category. Consider a structure \serif{C}\(^{op}\) with
	\begin{itemize}
		\item 
		Obj(\serif{C}\(^{op}\)) := Obj(\serif{C});
		\item
		for A, B objects of \serif{C}\(^{op}\) (hence objects of \serif{C}), \(\Hom_{\text{\serif{C}}^{op}}(A,B) \defeq \Hom_{\text{\serif{C}}}(B,A)\).
	\end{itemize}
	Show how to make this into a category (that is, define composition of morphisms in \serif{C}\(^{op}\) and verify the properties listed in \(\S 3.1\)).
	
	Intuitively, the ‘opposite’ category \serif{C}\(^{op}\) is simply obtained by ‘reversing all the arrows’ in \serif{C}. [\(5.1, \S \RN{8}.1.1, \S \RN{9}.1.2, \RN{9}.1.10\)]
\end{exercise}
\begin{solution}
	content...
\end{solution}

\begin{exercise}
	If \(A\) is a finite set, how large is \(\End_{\text{\serif{Set}}}(A)\)?
\end{exercise}
\begin{solution}
	content...
\end{solution}

\begin{exercise}
	\(\triangleright\) Formulate precisely what it means to say that \(1_a\) is an identity with respect to composition in Example 3.3, and prove this assertion. [\(\S 3.2\)]
\end{exercise}
\begin{solution}
	content...
\end{solution}

\begin{exercise}
	Can we define a category in the style of Example 3.3 using the relation \(<\) on the set \(\bZ\)?
\end{exercise}
\begin{solution}
	content...
\end{solution}

\begin{exercise}
	\(\triangleright\) Explain in what sense Example 3.4 is an instance of the categories considered in Example 3.3. [\(\S 3.2\)]
\end{exercise}
\begin{solution}
	content...
\end{solution}

\begin{exercise}
	\(\triangleright\) (Assuming some familiarity with linear algebra.) Define a category \serif{V} by taking Obj(\serif{V}) = \(\bN\) and letting \(\Hom_{\text{\serif{V}}}(n,m)\) = the set of \(n \times m\) matrices with real entries, for all \(n, m \in \bN\) (We will leave the reader the task of making sense of a matrix with 0 rows or columns.) Use product of matrices to define composition. Does this category ‘feel’ familiar? [\(\S \RN{6}.2.1, \S \RN{8}.1.3\)]
\end{exercise}
\begin{solution}
	content...
\end{solution}

\begin{exercise}
	\(\triangleright\) Define carefully objects and morphisms in Example 3.7, and draw the diagram
	corresponding to composition. [\(\S 3.2\)]
\end{exercise}
\begin{solution}
	content...
\end{solution}

\begin{exercise}
	\(\triangleright\) A \textit{subcategory} \serif{C'} of a category \serif{C} consists of a collection of objects of \serif{C}, with morphisms \(\Hom_{\text{\serif{C'}}}(A,B) \subseteq \Hom_{\text{\serif{C}}}(A,B)\) for all objects \(A, B\) in Obj(\serif{C'}), such that identities and compositions in \serif{C} make \serif{C'} into a category. A subcategory \serif{C'} is \textit{full} if \(\Hom_{\text{\serif{C'}}}(A,B) = \Hom_{\text{\serif{C}}}(A,B)\) for all \(A, B\) in Obj(\serif{C'}). Construct a category of \textit{infinite sets} and explain how it may be viewed as a full subcategory of \serif{Set}. [\(4.4, \S \RN{6}.1.1, \S \RN{8}.1.3\)]
\end{exercise}
\begin{solution}
	content...
\end{solution}

\begin{exercise}
	\(\triangleright\) An alternative to the notion of \textit{multiset} introduced in \(\S 2.2\) is obtained by considering sets endowed with equivalence relations; equivalent elements are taken to be multiple instances of elements ‘of the same kind’. Define a notion of morphism between such enhanced sets, obtaining a category \serif{MSet} containing (a ‘copy’ of) \serif{Set} as a full subcategory. (There may be more than one reasonable way to do this! This is intentionally an open-ended exercise.) Which objects in \serif{MSet} determine ordinary multisets as defined in \(\S 2.2\) and how? Spell out what a morphism of multisets would be from this point of view. (There are several natural notions of morphisms of multisets. Try to define morphisms in \serif{MSet} so that the notion you obtain for ordinary multisets captures your intuitive understanding of these objects.) [\(\S 2.2, \S 3.2, 4.5\)]
\end{exercise}
\begin{solution}
	content...
\end{solution}

\begin{exercise}
	Since the objects of a category \serif{C} are not (necessarily) sets, it is not clear how to make sense of a notion of ‘subobject’ in general. In some situations it \textit{does} make sense to talk about subobjects, and the subobjects of any given object \(A\) in \serif{C} are in one-to-one correspondence with the morphisms \(A \to \Omega\) for a fixed, special object \(\Omega\) of \serif{C}, called a \textit{subobject classifier}. Show that \serif{Set} has a subobject classifier.
\end{exercise}
\begin{solution}
	content...
\end{solution}

\begin{exercise}
	\(\triangleright\) Draw the relevant diagrams and define composition and identities for the category \serif{C}\(^{A,B}\) mentioned in Example 3.9. Do the same for the category \serif{C}\(^{\alpha,\beta}\) mentioned in Example 3.10. [\(\S 5.5, 5.12\)]
\end{exercise}
\begin{solution}
	content...
\end{solution}