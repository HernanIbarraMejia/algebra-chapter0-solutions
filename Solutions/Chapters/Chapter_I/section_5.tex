\section{Universal properties}
\extitle
	
\begin{exercise}
	Prove that a final object in a category \serif{C} is initial in the opposite category \serif{C\(^{\text{op}}\)}  (cf. Exercise 3.1).
\end{exercise}
\begin{solution}
	content...
\end{solution}

\begin{exercise}
	\(\triangleright\) Prove that \(\emptyset\) is the unique initial object in \serif{Set}. [\(\S 5.1\)]
\end{exercise}
\begin{solution}
	content...
\end{solution}

\begin{exercise}
	\(\triangleright\) Prove that final objects are \textit{unique} up to isomorphism. [\(\S 5.1\)]
\end{exercise}
\begin{solution}
	content...
\end{solution}

\begin{exercise}
	What are initial and final objects in the category of ‘pointed sets’ (Example 3.8)? Are they unique?
\end{exercise}
\begin{solution}
	content...
\end{solution}

\begin{exercise}
	\(\triangleright\) What are the final objects in the category considered in \(\S 5.3\)? [\(\S 5.3\)]
\end{exercise}
\begin{solution}
	content...
\end{solution}

\begin{exercise}
	\(\triangleright\) Consider the category corresponding to endowing (as in Example 3.3) the set \(\bZ^+\) of positive integers with the \textit{divisibility} relation. Thus there is exactly one morphism \(d \to m\) in this category if and only if \(d\) divides \(m\) without remainder; there is no morphism between \(d\) and \(m\) otherwise. Show that this category has products and coproducts. What are their ‘conventional’ names? [\(\S \RN{7}.5.1\)]
\end{exercise}
\begin{solution}
	content...
\end{solution}

\begin{exercise}
	Redo Exercise 2.9, this time using Proposition 5.4.
\end{exercise}
\begin{solution}
	content...
\end{solution}

\begin{exercise}
	Show that in every category \serif{C} the products \(A \times B\) and \(B \times A\) are isomorphic, if they exist. (Hint: Observe that they both satisfy the universal property for the product of \(A\) and \(B\); then use Proposition 5.4.)
\end{exercise}
\begin{solution}
	content...
\end{solution}

\begin{exercise}
	Let \serif{ C} be a category with products. Find a reasonable candidate for the	universal property that the product \(A \times B \times C\) of three objects of \serif{C} ought to satisfy, and prove that both \((A \times B) \times C\) and \(A \times (B \times C)\) satisfy this universal property. Deduce that \((A \times B) \times C\) and \(A \times (B \times C)\) are necessarily isomorphic.
\end{exercise}
\begin{solution}
	content...
\end{solution}

\begin{exercise}
	Push the envelope a little further still, and define products and coproducts for \textit{families} (i.e., indexed sets) of objects of a category.
	\begin{itemize}
		\item[] Do these exist in \serif{Set}?
		\item[] It is common to denote the product \(\underbrace{A \times \dots \times A}_{n \text{times}}\) by \(A^n\).
	\end{itemize}
\end{exercise}
\begin{solution}
	content...
\end{solution}

\begin{exercise}
	Let \(A\), resp. \(B\) be a set, endowed with an equivalence relation \(\sim_A\), resp. \(\sim_B\). Define a relation \(\sim\) on \(A \times B\) by setting
	\begin{equation*}
		(a_1, b_1) \sim (a_2, b_2) \iff a_1 \sim_A a_2 \text{ and } b_1 \sim_B b_2.
	\end{equation*}
	(This is immediately seen to be an equivalence relation.)
	\begin{itemize}
		\item Use the universal property for quotients (\(\S 5.3\)) to establish that there are functions
		\((A \times B)/{\sim} \to A/{\sim}_A\), \((A \times B)/{\sim} \to B/{\sim}_B\).
		\item Prove that \((A \times B)/{\sim}\), with these two functions, satisfies the universal property
		for the product of \(A/{\sim}_A\) and \(B/{\sim}_B\).
		\item Conclude (without further work) that \((A \times B)/{\sim} \cong (A/{\sim}_A) \times (B/{\sim}_B)\).
	\end{itemize}
\end{exercise}
\begin{solution}
	content...
\end{solution}

\begin{exercise}
	\(\neg\) Define the notions of \textit{fibered products} and \textit{fibered coproducts}, as terminal objects of the categories \serif{C}\(_{\alpha,\beta}\), \serif{C}\(^{\alpha,\beta}\) considered in Example 3.10 (cf. also Exercise 3.11), by stating carefully the corresponding universal properties.
	
	As it happens, \serif{Set} has both fibered products and coproducts. Define these objects ‘concretely’, in terms of naive set theory. [\(\RN{2}.3.9, \RN{3}.6.10, \RN{3}.6.11\)]
\end{exercise}
\begin{solution}
	content...
\end{solution}