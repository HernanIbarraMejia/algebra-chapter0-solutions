\section{Functions between sets}
\extitle

\begin{exercise}
	\(\triangleright\) How many different bijections are there between a set \(S\) with \(n\) elements and itself? [\(\S \RN{2}.2.1\)]
\end{exercise}
\begin{solution}
	content...
\end{solution}

\begin{exercise}
	\(\triangleright\) Prove statement (2) in Proposition 2.1. You may assume that given a family of disjoint subsets of a set, there is a way to choose one element in each member of the family. [\(\S 2.5, \RN{5}.3.3\)]
\end{exercise}
\begin{solution}
	content...
\end{solution}

\begin{exercise}
	Prove that the inverse of a bijection is a bijection and that the composition of two bijections is a bijection.
\end{exercise}
\begin{solution}
	content...
\end{solution}

\begin{exercise}
	\(\triangleright\) Prove that ‘isomorphism’ is an equivalence relation (on any set of sets). [\(\S 4.1\)]
\end{exercise}
\begin{solution}
	content...
\end{solution}

\begin{exercise}
	\(\triangleright\) Formulate a notion of \textit{epimorphism}, in the style of the notion of \textit{monomorphism} seen in \(\S 2.6\), and prove a result analogous to Proposition 2.3, for epimorphisms and surjections. [\(\S 2.6, \S 4.2\)]
\end{exercise}
\begin{solution}
	content...
\end{solution}

\begin{exercise}
	With notation as in Example 2.4, explain how any function \(f \colon A \to B\) determines a section of \(\pi_A\).
\end{exercise}
\begin{solution}
	content...
\end{solution}

\begin{exercise}
	Let \(f \colon A \to B\) be any function. Prove that the graph \(\Gamma_f\) of \(f\) is isomorphic to \(A\).
\end{exercise}
\begin{solution}
	content...
\end{solution}

\begin{exercise}
	Describe as explicitly as you can all terms in the canonical decomposition (cf. \(\S 2.8\)) of the function \(\bR \to \bC\) defined by \(r \mapsto e^{2\pi ir}\). (This exercise matches one assigned previously. Which one?)
\end{exercise}
\begin{solution}
	content...
\end{solution}

\begin{exercise}
	\(\triangleright\) Show that if \(A' \cong A''\) and \(B' \cong B''\), and further \(A' \cap B' = \emptyset\) and \(A'' \cap B'' = \emptyset\), then \(A' \cup B' \cong A'' \cup B''\). Conclude that the operation \(A \sqcup B\) (as described in \(\S 1.4\)) is well-defined up to \textit{isomorphism} (cf. \(\S 2.9\)). [\(\S 2.9, 5.7\)]
\end{exercise}
\begin{solution}
	content...
\end{solution}

\begin{exercise}
	\(\triangleright\) Show that if \(A\) and \(B\) are finite sets, then \(\ord{BA} = \ord{B}^{\ord{A}}\). [\(\S 2.1, 2.11, \S \RN{2}.4.1\)]
\end{exercise}
\begin{solution}
	content...
\end{solution}

\begin{exercise}
	\(\triangleright\) In view of Exercise 2.10, it is not unreasonable to use \(2^A\) to denote the set of functions from an arbitrary set \(A\) to a set with 2 elements (say \(\{0, 1\}\). Prove that there is a bijection between \(2^A\) and the \textit{power set} of \(A\) (cf. \(\S 1.2\)). [\(\S 1.2, \RN{3}.2.3\)]
\end{exercise}
\begin{solution}
	content...
\end{solution}