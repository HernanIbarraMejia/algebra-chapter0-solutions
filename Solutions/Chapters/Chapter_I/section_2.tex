\section{Functions between sets}
\extitle

\begin{exercise}
	\(\triangleright\) How many different bijections are there between a set \(S\) with \(n\) elements and itself? [\(\S \RNo{2}.2.1\)]
\end{exercise}
\begin{solution}
	$n!$.
\end{solution}

\begin{exercise}
	\(\triangleright\) Prove statement (2) in Proposition 2.1. You may assume that given a family of disjoint subsets of a set, there is a way to choose one element in each member of the family. [\(\S 2.5, \RNo{5}.3.3\)]
\end{exercise}
\begin{solution}
	First we prove that if $f\colon A \to B$ has a right inverse then it is a surjection. Let $g\colon B \to A$ the the right inverse of $f$. Let $b\in B$. Then clearly $f(g(b)) = b$, so $f$ is surjective.
	
	Conversely, assume $f$ is surjective. Then, if $b\in B$, we have that $f^{-1}(\{b\})$ is nonempty. For $b \in B$ we pick some $a_b\in f^{-1}(\{b\})$ and define $g\colon B \to A$ by $g(b)\coloneqq a_b$. By construction, $f\circ g = \textnormal{id}_B$.
\end{solution}

\begin{exercise}
	Prove that the inverse of a bijection is a bijection and that the composition of two bijections is a bijection.
\end{exercise}
\begin{solution}
	Let $f\colon A \to B$ and $g\colon B \to C$ be two bijections. Then $f^{-1}$ is also a bijection since it has a two-sided inverse, namely $f$. Also, $g\circ f$ is a bijection since $f^{-1} \circ g^{-1}$ is a two-sided inverse.
\end{solution}

\begin{exercise}
	\(\triangleright\) Prove that ‘isomorphism’ is an equivalence relation (on any set of sets). [\(\S 4.1\)]
\end{exercise}
\begin{solution}
	See Exercise 2.3 for some necessary results. Clearly for any set $A$ we have that $A$ is isomorphic to $A$ since $\id_A \colon A \to A$ is a bijection. If $A$ is isomorphic to $B$ then there is some bijection $f \colon A \to B$. But then $f^{-1}\colon B \to A$ is also a bijection and hence $B$ is isomorphic to $A$. Now suppose $A$ is isomorphic to $B$ and $B$ is isomorphic to $C$, where $f\colon A \to B$ and $g\colon B \to C$ are bijections. Then $g\circ f\colon A \to C$ is also a bijection and so $A$ is isomorphic to $C$.
\end{solution}

\begin{exercise}
	\(\triangleright\) Formulate a notion of \textit{epimorphism}, in the style of the notion of \textit{monomorphism} seen in \(\S 2.6\), and prove a result analogous to Proposition 2.3, for epimorphisms and surjections. [\(\S 2.6, \S 4.2\)]
\end{exercise}
\begin{solution}
	We say a function $f\colon A \to B$ is an \emph{epimorphism} (or \emph{epic}) if the following holds:
	\[
		\textnormal{\textit{for all sets $Z$ and all functions $\beta',\beta''\colon B \to Z$}}
	\]
	\[
		\beta'\circ f = \beta'' \circ f \implies \beta' = \beta''.
	\]
	Now we claim that a function is an epimorphism iff it is a surjection. Indeed, if $f\colon A \to B$ is a surjection then it has a right inverse $g\colon B \to A$. Suppose $Z$ is a set and let $\beta', \beta''\colon B \to Z$ be functions with $\beta' \circ f = \beta'' \circ f$. Then $(\beta' \circ f)\circ g = (\beta'' \circ f)\circ g$ which implies $\beta' \circ (f\circ g) = \beta'' \circ (f\circ g)$, and this in turn implies $\beta' \circ \id_B = \beta'' \circ \id_B$, and so $\beta = \beta''$ as desired.
	
	Now suppose $f\colon A \to B$ is an epimorphism. For the sake of contradiction, suppose there is some $b_0\in B$ such that $f(a) \neq b_0$ for all $a\in A$. Let $Z = \{0,1\}$ and define $\beta',\beta''\colon B \to Z$ by
	\[
		\beta'(b) \coloneqq 0 \,\textnormal{ and } \,\beta''(b)\coloneqq \begin{cases}
			0 & \textnormal{if } b\neq b_0\\
			1 & \textnormal{if } b = b_0
		\end{cases},
	\]
	for all $b\in B$. Clearly $\beta'\circ f = \beta'' \circ f$ but $\beta' \neq \beta''$, a contradiction. So $f$ must be a surjection.
\end{solution}

\begin{exercise}
	With notation as in Example 2.4, explain how any function \(f \colon A \to B\) determines a section of \(\pi_A\).
\end{exercise}
\begin{solution}
	Define $\pi_A^*\colon A \to A \times B$ by the rule $\pi_A^*(a) \coloneqq (a, f(a))$. This is manifestly a section of $\pi_A$.
\end{solution}

\begin{exercise}
	Let \(f \colon A \to B\) be any function. Prove that the graph \(\Gamma_f\) of \(f\) is isomorphic to \(A\).
\end{exercise}
\begin{solution}
	Define $f^*\colon A \to \Gamma_f$ by the rule $f^*(a) \coloneqq (a,f(a))$. We claim that $f^*$ is a bijection. Indeed, we will see that the natural projection restricted to $\Gamma_f$, written as $\restr{\pi_A}{\Gamma_f}$, is a two-sided inverse. Let $a\in A$ and consider
	\begin{align*}
		(\restr{\pi_A}{\Gamma_f} \circ f^*)(a) &= \restr{\pi_A}{\Gamma_f}(a,f(a))\\
		&= a = \id_A(a).
	\end{align*}
	Similarly, let $(a,f(a))$ be an arbitrary element of $\Gamma_f$. Then
	\begin{align*}
		(f^*\circ \restr{\pi_A}{\Gamma_f})(a,f(a)) &= f^{*}(a)\\
		&= (a,f(a)) = \id_{\Gamma_f}(a,f(a)).
	\end{align*}
	Thus $f^*$ is a bijection.
\end{solution}

\begin{exercise}
	Describe as explicitly as you can all terms in the canonical decomposition (cf. \(\S 2.8\)) of the function \(\bR \to \bC\) defined by \(r \mapsto e^{2\pi ir}\). (This exercise matches one assigned previously. Which one?)
\end{exercise}
\begin{solution}
	Let $f \colon \bR \to \bC$ be defined by $f(r) \coloneqq e^{2\pi i r}$. We define an equivalence relation on $\bR$ by $a\sim a' \iff f(a') = f(a'')$. This is easily seen to be the same as the equivalence relation defined in Exercise 1.6. In that exercise we saw that the quotient can be identified with the interval $[0,1)$ and the projection $\pi \colon \bR \to \bR/\sim$ is assigning to each real number its (positive) fractional part. Then the canonical decomposition gives a bijection from the quotient, i.e. $[0,1)$, to the image of $f$, i.e. the unit circle in the complex plane, by assigning to each $x \in [0,1)$ the point $e^{2\pi ix}$. Finally, this unit circle is included in the whole complex plane, in the obvious sense.
\end{solution}

\begin{exercise}
	\(\triangleright\) Show that if \(A' \cong A''\) and \(B' \cong B''\), and further \(A' \cap B' = \emptyset\) and \(A'' \cap B'' = \emptyset\), then \(A' \cup B' \cong A'' \cup B''\). Conclude that the operation \(A \sqcup B\) (as described in \(\S 1.4\)) is well-defined up to \textit{isomorphism} (cf. \(\S 2.9\)). [\(\S 2.9, 5.7\)]
\end{exercise}
\begin{solution}
	Let $f\colon A' \to A''$ and $g\colon B \to B'$ be bijections. Let us define $f\oplus g \colon A'\cup B' \to A''\cup B''$ by the rule
	\[
		f\oplus g(x) \coloneqq 
		\begin{cases}
			f(x) & \textnormal{if } x\in A'\\
			g(x) & \textnormal{if } x\in B'
		\end{cases}.
	\]
	This is well defined since if $x\in A'\cup B'$ then $x\in A'$ or $x\in B'$ but not both, since $A'\cap B' = \emptyset$.
	
	Now we prove $f\oplus g$ is a bijection. Define $f^{-1} \oplus g^{-1}\colon A''\cup B'' \to A' \cup B'$ by 
	\[
		f^{-1}\oplus g^{-1} (y)\coloneqq 
		\begin{cases}
			f^{-1}(y) & \textnormal{if } y \in A''\\
			g^{-1}(y) & \textnormal{if } y \in B''
		\end{cases}.
	\]
	This is well defined since if $y\in A''\cup B''$ then $y\in A''$ or $y\in B''$ but not both, since $A''\cap B'' = \emptyset$. It is immediately verified that $f\oplus g$ and $f^{-1}\oplus g^{-1}$ are inverses of each other and hence they are bijections.
	
	In conclusion, no matter how we make disjoint copies of $A$ and $B$, the resulting disjoint unions will be isomorphic. Hence, it makes some sense to talk about \emph{the} disjoint union of $A$ and $B$.
\end{solution}

\begin{exercise}
	\(\triangleright\) Show that if \(A\) and \(B\) are finite sets, then \(\ord{B^A} = \ord{B}^{\ord{A}}\). [\(\S 2.1, 2.11, \S \RNo{2}.4.1\)]
\end{exercise}
\begin{solution}
	Let $|A| = n$. We use induction on $n$. If $n = 0$ then $A = \emptyset$ and there is only one function $\emptyset \to B$, and further $1 = |B|^0$. This closes the base case. 
	
	Suppose the claim is true for some natural number $n$. Defining a function from a set of $n+1$ elements to $B$ is the same as first defining it for $n$ elements, for which there are $|B|^n$ choices by inductive hypothesis, and then figuring out where the last element goes, for which there are $|B|$ choices. Overall, there must be $|B|^n|B| = |B|^{n+1}$ ways of defining the function when $|A| = n+1$. This closes the induction.
\end{solution}

\begin{exercise}
	\(\triangleright\) In view of Exercise 2.10, it is not unreasonable to use \(2^A\) to denote the set of functions from an arbitrary set \(A\) to a set with 2 elements (say \(\{0, 1\}\). Prove that there is a bijection between \(2^A\) and the \textit{power set} of \(A\) (cf. \(\S 1.2\)). [\(\S 1.2, \RNo{3}.2.3\)]
\end{exercise}
\begin{solution}
	Define $F\colon 2^A \to \mathscr{P}(A)$ by the rule
	\[
		F(f) \coloneqq f^{-1}(\{1\}),\textnormal{ for all }f\colon A \to \{0,1\}.
	\]
	Now define $G\colon \mathscr{P}(A) \to 2^A$ by saying that, for all $S \subseteq A$ we have
	\[
		G(S) \coloneqq \mathbf{1}_S,
	\]
	where $\mathbf{1}_S$ is the indicator function of $S$, defined on $A$. It is readily seen that $F$ and $G$ are inverses of each other and hence bijections.
\end{solution}