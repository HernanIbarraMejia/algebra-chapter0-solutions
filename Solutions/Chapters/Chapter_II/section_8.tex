\section{Canonical decomposition and Lagrange's theorem}
\extitle

\begin{exercise}
	If a group $H$ may be realized as a subgroup of two groups $G_1$ and $G_2$ and if 
	\[
		\frac{G_1}{H} \cong \frac{G_2}{H},
	\]
	does it follow that $G_1 \cong G_2$? Give a proof or a counterexample.
\end{exercise}
\begin{solution}
	content...
\end{solution}

\begin{exercise}
	$\neg$ Extend Example 8.6 as follows. Suppose $G$ is a group and $H\subseteq G$ is a subgroup of \emph{index} 2, that is, such that there are precisely two (say, left-) cosets of $H$ in $G$. Prove that $H$ is normal in $G$. [9.11, $\RNo{4}.1.16$]
\end{exercise}
\begin{solution}
	content...
\end{solution}

\begin{exercise}
	Prove that every finite group is finitely presented.
\end{exercise}
\begin{solution}
	content...
\end{solution}

\begin{exercise}
	Prove that $(a,b|a^2,b^2,(ab)^n)$ is a presentation of the dihedral group $D_{2n}$. (Hint: With respect to the generators defined in Exercise 2.5, set $a = x$ and $b = xy$; prove you can get the relations given here from the ones you obtained in Exercise 2.5, and conversely.)
\end{exercise}
\begin{solution}
	content...
\end{solution}

\begin{exercise}
	Let $a,b$ be distinct elements of order 2 in a group $G$, and assume that $ab$ has finite order $n\geq 3$. Prove that the subgroup generated by $a$ and $b$ in $G$ is isomorphic to the dihedral group $D_{2n}$. (Use the previous exercise.)
\end{exercise}
\begin{solution}
	content...
\end{solution}

\begin{exercise}
	$\neg$ Let $G$ be a group, and let $A$ be a set of generators for $G$; assume $A$ is finite. The corresponding \emph{Cayley graph} is a directed graph whose set of vertices is in one-to-one correspondence with $G$, and two vertices $g_1$, $g_2$ are connected by an edge if $g_2 = g_1 a$ for an $a\in A$; this edge may be labeled $a$ an oriented from $g_1$ to $g_2$. For example, the graph drawn in Example 5.3 for the free group $F(\{x,y\})$ on two generators $x$, $y$ is the corresponding Cayley graph (with the convention that horizontal edges are labeled $x$ and point to the right and vertical edges are labeled $y$ and point up).
	
	Prove that if a Cayley graph of a group is a tree, then the group is free. Conversely, prove that free groups admit Cayley graphs that are trees. [$\S 5.3$, 9.15]
\end{exercise}
\begin{solution}
	content...
\end{solution}

\begin{exercise}
	$\triangleright$ Let $(A|\mathscr{R})$ , resp., $(A'|\mathscr{R}')$ be a presentation for a group $G$, resp., $G'$ (cf. $\S 8.2$); we may assume that $A, A'$ are disjoint. Prove that the group $G \ast G'$ presented by
	\[
		(A \cup A'|\mathscr{R} \cup \mathscr{R}')
	\]
	satisfies the universal property for the \emph{coproduct} of $G$ and $G'$ in $\srf{Grp}$. (Use the universal properties of both free groups and quotients to construct natural homomorphisms $G \to G\ast G'$, $G'\to G \ast G'$.) [$\S 3.4$, $\S 8.2$, 9.14]
\end{exercise}
\begin{solution}
	content...
\end{solution}

\begin{exercise}
	$\neg$ (If you know about matrices (cf. Exercise 6.1).) Prove that $\SL_n(\mathbb{R})$ is a \emph{normal subgroup} of $\GL_n(\mathbb{R})$, and `compute' $\GL_n(\mathbb{R})/\SL_n(\mathbb{R})$ as a well-known group. [$\RNo{6}$.3.3]
\end{exercise}
\begin{solution}
	content...
\end{solution}

\begin{exercise}
	$\neg$ (Ditto.) Prove that $\SO_n(\mathbb{R})\cong \SU(2)/{\pm I_2}$, where $I_2$ is the identity matrix. (Hint: It so happens that every matrix in $\SO_3(\mathbb{R})$ can be written in the form
	\[
		\begin{pmatrix}
			a^2 + b^2 - c^2 - d^2 & 2(bc-ad) & 2(ac + bd)\\
			2(ad + bc) & a^2 - b^2 + c^2 - d^2 & 2(cd - ab)\\
			2(bd - ac) & 2(ab+cd) & a^2 - b^2 - c^2 + d^2
		\end{pmatrix}
	\]
	where $a,b,c,d\in \mathbb{R}$ and $a^2 + b^2 + c^2 + d^2 = 1$. Proving this fact is not hard, but at this stage you will probably find it computationally demanding. Feel free to assume this, and use Exercise 6.3 to construct a surjective homomorphism $\SU(2)\to \SO_3(\mathbb{R})$; compute the kernel of this homomorphism.)
	
	If you know a little topology, you can now conclude that the fundamental group of $\SO_3(\mathbb{R})$ is $C_2$. [9.1, $\RNo{6}.1.3$]
\end{exercise}
\begin{solution}
	content...
\end{solution}

\begin{exercise}
	View $\mathbb{Z}\times \mathbb{Z}$ as a subgroup of $\mathbb{R}\times \mathbb{R}$:
	\[
	\begin{tikzpicture}[>=Stealth, dot/.style = {shape= circle, draw, fill = black, inner sep= 0pt, minimum size=2pt}]
		\draw[->] (-1,0) -- (1.5,0);
		\draw[->] (0,-1) -- (0,1.5);
		\foreach \x in {-1,-0.5,...,1}
		{
			\foreach \y in {-1,-0.5,...,1}
			{
				\node at (\x,\y) [dot] {};
			}
		}
	\end{tikzpicture} 
	\]
	Describe the quotient
	\[
		\frac{\mathbb{R}\times \mathbb{R}}{\mathbb{Z}\times \mathbb{Z}}
	\]
	in terms analogous to those used in Example 8.7. (Can you `draw a picture' of this group? Cf. Exercise $\RNo{1}.1.6$.)
\end{exercise}
\begin{solution}
	content...
\end{solution}

\begin{exercise}
	(Notation as in Proposition 8.10.) Prove `by hand' (that is, without invoking universal properties) that $N$ is normal in $G$ if and only if $N/H$ is normal in $G/H$.
\end{exercise}
\begin{solution}
	content...
\end{solution}

\begin{exercise}
	(Notation as in Proposition 8.11.) Prove `by hand' (that is, without invoking universal properties) that $HK$ is a subgroup of $G$ if $H$ is normal.
\end{exercise}
\begin{solution}
	content...
\end{solution}

\begin{exercise}
	$\neg$ Let $G$ be a finite group, and assume $|G|$ is odd. Prove that every element of $G$ is a square. [8.14]
\end{exercise}
\begin{solution}
	content...
\end{solution}

\begin{exercise}
	Generalize the result of Exercise 8.13: if $G$ is a group of order $n$ and $k$ is an integer relatively prime to $n$, then the function $G \to G$, $g\mapsto g^k$ is surjective.
\end{exercise}
\begin{solution}
	content...
\end{solution}

\begin{exercise}
	Let $a$, $n$ be positive integers, with $a>1$. Prove that $n$ divides $\phi(a^n -1)$, where $\phi$ is Euler's $\phi$-function; see Exercise 6.14. (Hint: Example 8.15.)
\end{exercise}
\begin{solution}
	content...
\end{solution}

\begin{exercise}
	Generalize Fermat's little theorem to congruence module arbitrary (that is, possibly nonprime) integers. Note that it is \emph{not} true that $a^n \equiv a \mod n$ for all $a$ and $n$: for example, $2^4$ is not congruent to $2$ modulo 4. \emph{What} is true? (This generalization is known as \emph{Euler's theorem}.)
\end{exercise}
\begin{solution}
	content...
\end{solution}

\begin{exercise}
	$\triangleright$ Assume $G$ is a finite abelian group, and let $p$ be a prime divisor of $|G|$. Prove that there exists an element in $G$ of order $p$. (Hint: Let $g\neq e$ be an element of $G$, and consider the subgroup $\langle g \rangle$; use the fact that this subgroup is cyclic to show that there is an element $h\in \langle g \rangle$ in $G$ of \emph{prime} order $q$. If $q = p$, you are done; otherwise, use the quotient $G/\langle h \rangle$ and induction.) [$\S 8.5$, 8.18, 8.20, $\RNo{4}. 2.1$]
\end{exercise}
\begin{solution}
	content...
\end{solution}

\begin{exercise}
	Let $G$ be an abelian group of order $2n$, where $n$ is odd. Prove that $G$ has \emph{exactly one} element of order 2. (It has at least one, for example by Exercise 8.17. Use Lagrange's theorem to establish that it cannot have more than one.) Does the same conclusion hold if $G$ is not necessarily commutative?
\end{exercise}
\begin{solution}
	content...
\end{solution}

\begin{exercise}
	Let $G$ be a finite group, and let $d$ be a proper divisor of $|G|$. Is it necessarily true that there exists an element of $G$ of order $d$? Give a proof or a counterexample.
\end{exercise}
\begin{solution}
	content...
\end{solution}

\begin{exercise}
	$\triangleright$ Assume $G$ is a finite abelian group, and let $d$ be a divisor of $|G|$. Prove that there exists a \emph{subgroup} $H\subseteq G$ of order $d$. (Hint: induction; use Exercise 8.17.) [$\S \RNo{4}.2.2$]
\end{exercise}
\begin{solution}
	content...
\end{solution}

\begin{exercise}
	$\triangleright$ Let $H$, $K$ be subgroups of a group $G$. Construct a bijection between the set of cosets $hK$ with $h\in H$ and the set of left-cosets of $H\cap K$ in $H$. If $H$ and $K$ are finite, prove that 
	\[
		|HK| = \frac{|H|\cdot|K|}{|H\cap K|}
	\]
\end{exercise}
\begin{solution}
	content...
\end{solution}

\begin{exercise}
	$\triangleright$ Let $\varphi\colon G \to G'$ be a group homomorphism, and let $N$ be the smallest normal subgroup containing $\im \varphi$. Prove that $G'/N$ satisfies the universal property of $\coker \varphi$ in $\srf{Grp}$. [$\S 8.6$]
\end{exercise}
\begin{solution}
	content...
\end{solution}

\begin{exercise}
	$\triangleright$ Consider the subgroup 
	\[
		H = 
		\left\{
		\begin{pmatrix}
			1 & 2& 3 \\
			1 & 2 & 3
		\end{pmatrix}
		\begin{pmatrix}
			1 & 2& 3 \\
			2 & 1 & 3
		\end{pmatrix}
		\right\}
	\]
	of $S_3$. Show that the cokernel of the inclusion $H \hookrightarrow S_3$ is trivial, although $H \hookrightarrow S_3$ is not surjective. [$\S 8.6$]
\end{exercise}
\begin{solution}
	content...
\end{solution}

\begin{exercise}
	$\triangleright$ Show that epimorphisms in $\srf{Grp}$ do not necessarily have right-inverses. [$\S \RNo{1}. 4.2$]
\end{exercise}
\begin{solution}
	content...
\end{solution}

\begin{exercise}
	Let $H$ be a commutative normal subgroup of $G$. Construct an interesting homomorphism from $G/ H$ to $\Aut(H)$. (Cf. Exercise 7.10.)
\end{exercise}
\begin{solution}
	content...
\end{solution}





