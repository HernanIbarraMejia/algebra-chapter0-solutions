\section{Free groups}
\extitle
\begin{exercise}
	Does the category $\mathscr{F}^A$ defined in $\S 5.2$ have final objects? If so, what are they?
\end{exercise}
\begin{solution}
	content...
\end{solution}

\begin{exercise}
	Since trivial groups $T$ are initial in $\srf{Grp}$, one may be led to think that $(e,T)$ should be initial in $\mathscr{F}^A$, for every $A$, $e$ would be defined by sending every element of $A$ to the (only) element in $T$; and for any other group $G$, \emph{there is a unique} homomorphism $T \to G$. Explain why $(e,T)$ is \emph{not} initial in $\mathscr{F}^A$ (unless $A =\emptyset$).
\end{exercise}
\begin{solution}
	content...
\end{solution}

\begin{exercise}
	$\triangleright$ Use the universal property of free groups to prove that the map $j\colon A \to F(A)$ is injective, for all sets $A$. (Hint$\colon$ It suffices to show that for every two elements $a$, $b$ of $A$ there is a group $G$ and a set-function $f\colon A \to G$ such that $f(a) \neq f(b)$. Why? How do you construct $f$ and $G$?) [$\S\RNo{3}.6.3$]
\end{exercise}
\begin{solution}
	content...
\end{solution}

\begin{exercise}
	In the `concrete' construction of free groups, one can try to reduce words by performing cancellations in any order; the process of `elementary reductions' used in the text (that is, from left to right) is only one possibility. Prove that the result of iterating cancellations on a word is independent of the order in which the cancellations are performed. Deduce the associativity of the product in $F(A)$ from this. [$\S5.3$]
\end{exercise}
\begin{solution}
	content...
\end{solution}

\begin{exercise}
	Verify explicitly that $H^{\oplus A}$ is a group. 
\end{exercise}
\begin{solution}
	content...
\end{solution}

\begin{exercise}
	$\triangleright$ Prove that the group $F(\{x,y\})$ (visualized in Example 5.3) is a coproduct $\bZ\ast \bZ$ of $\bZ$ by itself in the category $\srf{Grp}$. (Hint: With due care, the universal property for one turns into the universal property for the other.) [$\S 3.4$, $3.7$, $5.7$]
\end{exercise}
\begin{solution}
	content...
\end{solution}

\begin{exercise}
	$\triangleright$ Extend the result of Exercise 5.6 to free groups $F(\{x_1,\ldots, x_n\})$ and to free abelian groups $F^{ab}(\{x_1,\ldots,x_n\})$. [$\S 3.4$, $\S 5.4$]
\end{exercise}
\begin{solution}
	content...
\end{solution}

\begin{exercise}
	Still more generally, prove that $F(A \amalg B) = F(A) \ast F(B)$ and that $F^{ab}(A\amalg B) = F^{ab}(A)\oplus F^{ab}(B)$ for all sets $A$, $B$. (That is, the constructions $F$, $F^{ab}$ `preserve coproducts'.) 
\end{exercise}
\begin{solution}
	content...
\end{solution}

\begin{exercise}
	Let $G = \bZ^{\oplus\bN}$. Prove that $G\times G \cong G$.
\end{exercise}
\begin{solution}
	content...
\end{solution}

\begin{exercise}
	$\neg$ Let $F = F^{ab}(A)$.
	\begin{itemize}[leftmargin=16pt]%not sure what margins he uses!
		\item Define an equivalence relation $\sim$ on $F$ by setting $f'\sim f$ if and only if $f-f' = 2g$ for some $g\in F$. Prove that $F/{\sim}$ is a finite set if and only if $A$ is finite, and in that case $|F/{\sim} = 2^{|A|}$.
		\item Assume $F^{ab}(B) \cong F^{ab}(A)$. If $A$ is finite, prove that $B$ is also, and that $A \cong B$ as sets. (This results holds for free groups as well, and without any finiteness hypothesis. See Exercises 7.13 and $\RNo{6}.1.20$.)
	\end{itemize}
	[7.4, 7.13]
\end{exercise}