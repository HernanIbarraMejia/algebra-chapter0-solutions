\section{Quotient groups}
\extitle

\begin{exercise}
	$\triangleright$ List all subgroups of $S_3$ (cf. Exercise 6.13) and determine which subgroups are normal and which are not normal. [$\S 7.1$]
\end{exercise}
\begin{solution}
	content...
\end{solution}

\begin{exercise}
	Is the \emph{image} of a group homomorphism necessarily a \emph{normal} subgroup of the target?
\end{exercise}
\begin{solution}
	content...
\end{solution}

\begin{exercise}
	$\triangleright$ Verify that the equivalent conditions for normality given in $\S 7.1$ are indeed equivalent.
\end{exercise}
\begin{solution}
	content...
\end{solution}

\begin{exercise}
	Prove that the relation defined in Exercise 5.10 on a free abelian group $F = F^{ab}(A)$ is compatible with the group structure. Determine the quotient $F/{\sim}$ as a better known group.
\end{exercise}
\begin{solution}
	content...
\end{solution}

\begin{exercise}
	$\neg$ Define an equivalence relation $\sim$ on $\SL_2(\mathbb{Z})$ by letting $A\sim A' \iff A' = \pm A$. Prove that $\sim$ is compatible with the group structure. The quotient $\SL_2(\mathbb{Z})/{\sim}$ is denoted $\PSL_2(\mathbb{Z})$ and is called the \emph{modular group}; it would be a serious contender in a contest for `the most important group in mathematics', due to its role in algebraic geometry and number theory. Prove that $\PSL_2(\mathbb{Z})$ is generated by the (cosets of the) matrices
	\[
		\begin{pmatrix*}[r]
			0 & -1 \\
			1 & 0
		\end{pmatrix*}
		\;\;
		\textnormal{ and }
		\;\;
		\begin{pmatrix*}[r]
			1 & 1 \\
			0 & 1
		\end{pmatrix*}.
	\]
	(You will not need to work very hard, if you use the result of Exercise 6.10.) Note that the first has order 2 in $\PSL_2(\mathbb{Z})$, the second one has order 3, and their product has infinite order. [9.14]
\end{exercise}
\begin{solution}
	content...
\end{solution}

\begin{exercise}
	Let $G$ be a group, and let $n$ be a positive integer. Consider the relation
	\[
		a \sim b \iff (\exists g \in G) ab^{-1} = g^n.
	\]
	\begin{itemize}
		\item Show that in general $\sim$ is \emph{not} an equivalence relation.
		\item Prove that $\sim$ is an equivalence relation if $G$ is commutative, and determine the corresponding subgroup of $G$.
	\end{itemize}
\end{exercise}
\begin{solution}
	content...
\end{solution}

\begin{exercise}
	Let $G$ be a group, $n$ a positive integer, and let $H\subseteq G$ be the subgroup generated by all elements of order $n$ in $G$. Prove that $H$ is normal.
\end{exercise}
\begin{solution}
	content...
\end{solution}

\begin{exercise}
	$\triangleright$ Prove Proposition 7.6. [$\S 7.3$]
\end{exercise}
\begin{solution}
	content...
\end{solution}

\begin{exercise}
	State and prove `mirror' statements of Proposition 7.4 and 7.6, leading to the description of relations satisfying ($\dagger \dagger$).
\end{exercise}
\begin{solution}
	content...
\end{solution}

\begin{exercise}
	$\neg$ Let $G$ be a group, and $H \subseteq G$ a subgroup. With notation as in Exercise 6.7, show that $H$ is normal in $G$ if and only if $\forall\gamma \in \Inn(G)$, $\gamma(H)\subseteq H$.
	
	Conclude that if $H$ is normal in $G$, then there is an interesting homomorphism $\Inn(G) \to \Aut(H)$. [8.25]
\end{exercise}
\begin{solution}
	content...
\end{solution}

\begin{exercise}
	$\triangleright$ Let $G$ be a group, and let $[G,G]$ be the subgroup of $G$ generated by all elements of the form $aba^{-1}b^{-1}$. (This is the \emph{commutator} subgroup of $G$; we will return to it in $\S \RNo{4}.3.3$.) Prove that $[G,G]$ is normal in $G$. (Hint: With notation as in Exercise 4.8, $g\cdot aba^{-1}b^{-1} = \gamma_g(aba^{-1}b^{-1})$.) Prove that $G/[G,G]$ is commutative. [7.12, $\S\RNo{4}.3.3$]
\end{exercise}
\begin{solution}
	content...
\end{solution}

\begin{exercise}
	$\triangleright$ Let $F = F(A)$ be a free group, and let $f\colon A \to G$ be a set-function from the set $A$ to a \emph{commutative} group $G$. Prove that $f$ induces a unique homomorphism $F/[F,F]\to G$, where $[F,F]$ is the commutator subgroup of $F$ defined in Exercise 7.11. (Use Theorem 7.12.) Conclude that $F/[F,F]\cong F^{ab}(A)$. (Use Proposition I.5.4.) [$\S 6.4$, 7.13, $\RNo{6}.1.20$]
\end{exercise}
\begin{solution}
	content...
\end{solution}

\begin{exercise}
	$\neg$ Let $A,B$ be sets and $F(A)$, $F(B)$ the corresponding free groups. Assume $F(A) \cong F(B)$. If $A$ is finite, prove that $B$ is also and $A\cong B$. (Use Exercise 7.12 to upgrade Exercise 5.10.) [5.10, $\RNo{6}.1.20$]
\end{exercise}
\begin{solution}
	content...
\end{solution}

\begin{exercise}
	Let $G$ be a group. Prove that $\Inn(G)$ is a \emph{normal} subgroup of $\Aut(G)$.
\end{exercise}
\begin{solution}
	content...
\end{solution}