\section{Examples of groups}
\extitle

\begin{exercise}
	$\neg$ One can associate an $n\times n$ matrix $M_{\sigma}$ with a permutation $\sigma\in S_n$ by letting the entry at $(i,(i)\sigma)$ be 1 and letting all other entries be 0. For example, the matrix corresponding to the permutation
	\[
		\sigma = 
		\begin{pmatrix}
			1 & 2 & 3 \\
			3 & 1 & 2
		\end{pmatrix}
		\in S_n
	\]
	 would be 
	 \[
	 	M_\sigma = 
	 	\begin{pmatrix}
	 		0 & 0 & 1\\
	 		1 & 0 & 0\\
	 		0 & 1 & 0 
	 	\end{pmatrix}.
	 \]
	 Prove that, with this notation, 
	 \[M_{\sigma\tau} = M_\sigma M_\tau\]
	 for all $\sigma, \tau\in S_n$, where the product on the right is the ordinary product of matrices. [\RNo{4}.4.13]
\end{exercise}
\begin{solution}
	content...
\end{solution}

\begin{exercise}
	$\triangleright$ Prove that if $d\leq n$, then $S_n$ contains elements of order $d$. [$\S 2.1$]
\end{exercise}
\begin{solution}
	content...
\end{solution}

\begin{exercise}
	For every positive integer $n$, find an element of order $n$ in $S_{\bN}$.
\end{exercise}
\begin{solution}
	content...
\end{solution}

\begin{exercise}
	Define a homomorphism $D_8 \to S_4$ by labeling vertices of a square, as we did for a triangle in $\S 2.2$. List the 8 permutations in the image of this homomorphism.
\end{exercise}
\begin{solution}
	content...
\end{solution}

\begin{exercise}
	$\triangleright$ Describe the generators and relations for all dihedral groups $D_{2n}$. (Hint: Let $x$ be the reflection about a line through the center of a regular $n$-gon and a vertex, and let $y$ be the counterclockwise rotation by $2\pi/n$. The group $D_{2n}$ will be generated by $x$ and $y$ subject to three relations. To see that these relations really determine $D_{2n}$, use them to show that any product $x^{i_1} y^{i_2}x^{i_3}y^{i_4}\cdots$ equals $x^iy^j$ for some $i$, $j$ with $0 \leq i \leq 1$, $0\leq j < n$.)[8.4, $\RNo{4}.2.5$]
\end{exercise}
\begin{solution}
	content...
\end{solution}

\begin{exercise}
	$\triangleright$ For every positive integer $n$ construct a group containing elements $g$, $h$ such that $|g| = 2$, $|h| = 2$, and $|gh|= n$. (Hint: For $n > 1$, $D_{2n}$ will do.) [$\S 1.6$] 
\end{exercise}
\begin{solution}
	content...
\end{solution}

\begin{exercise}
	$\neg$ Find all elements of $D_{2n}$ that commute with every other element. (The parity of $n$ plays a role.) [\RNo{4}.1.2]
\end{exercise}
\begin{solution}
	content...
\end{solution}

\begin{exercise}
	Find the orders of the groups of symmetries of the five `platonic solids'.
\end{exercise}
\begin{solution}
	content...
\end{solution}

\begin{exercise}
	Verify carefully that `congruence mod $n$' is an equivalence relation.
\end{exercise}
\begin{solution}
	content...
\end{solution}

\begin{exercise}
	Prove that if $n>0$, then $\bZ/n\bZ$ consists of precisely $n$ elements.
\end{exercise}
\begin{solution}
	content...
\end{solution}

\begin{exercise}
	$\triangleright$ Prove that the square of every odd integer is congruent to 1 modulo 8. [$\RNo{7}.5.1$]
\end{exercise}
\begin{solution}
	content...
\end{solution}

\begin{exercise}
	Prove that there are no nonzero integers $a,b,c$ such that $a^2 + b^2 = 3c^2$. (Hint: By studying the equation $[a]_4^2 + [b]_4^2 = 3[c]_4^2$ in $\bZ/4\bZ$, show that $a$, $b$, $c$ would all have to be even. Letting $a = 2k$, $b= 2l$, $c = 2m$, you would have $k^2 + l^2 = 3m^2$. What's wrong with that?)
\end{exercise}
\begin{solution}
	content...
\end{solution}

\begin{exercise}
	$\triangleright$ Prove that if $\gcd(m,n) = 1$, then there exist integers $a$ and $b$ such that 
	\[
		am + bn = 1.
	\]
	(Use Corollary 2.5.) Conversely, prove that if $am+bn = 1$ for some integers $a$ and $b$, then $\gcd(m,n) = 1$. [2.15, $\S \RNo{5}.2.1$, $\RNo{5}.2.4$]
\end{exercise}
\begin{solution}
	content...
\end{solution}

\begin{exercise}
	$\triangleright$ State and prove an analog of Lemma 2.2, showing that the multiplication on $\bZ/n\bZ$ is a well-defined operation. [$\S 2.3$, $\S \RNo{3}$.1.2]
\end{exercise}
\begin{solution}
	content...
\end{solution}

\begin{exercise}
	$\neg$ Let $n > 0$ be an odd integer.
	\begin{itemize}
		\item Prove that if $\gcd(m,n) = 1$, then $\gcd(2m+n, 2n) = 1$. (Use Exercise 2.13.)
		\item Prove that if $\gcd(r,2n) = 1$, then $\gcd(\frac{r-n}{2}, n ) = 1$. (Ditto.)
		\item Conclude that the function $[m]_n \to [2m + n]_{2n}$ is a bijection between $(\bZ/n\bZ)^*$ and $(\bZ/2n\bZ)^*$.
	\end{itemize}
	The number $\phi(n)$ of elements of $(\bZ/n\bZ)^*$ is \emph{Euler's $\phi$-function}. The reader has just proved that if $n$ is odd, then $\phi(2n) = \phi(n)$. Much more general formulas will be given later on (cf. Exercise $\RNo{5}.6.8$). [$\RNo{7}.5.11$]
\end{exercise}
\begin{solution}
	content...
\end{solution}

\begin{exercise}
	Find the last digit of $1238237^{18238456}$. (Work in $\bZ/10\bZ$.)
\end{exercise}
\begin{solution}
	content...
\end{solution}

\begin{exercise}
	$\triangleright$ Show that if $m\equiv m'$ mod $n$, then $\gcd(m,n) = 1$ if and only if $\gcd(m',n) = 1$. [$\S 2.3$]
\end{exercise}
\begin{solution}
	content...
\end{solution}

\begin{exercise}
	For $d\leq n$, define an injective function $\bZ/d\bZ \to S_n$ preserving the operation, that is, such that the sum of equivalence classes in $\bZ/d\bZ$ corresponds to the product of the corresponding permutations.
\end{exercise}
\begin{solution}
	content...
\end{solution}

\begin{exercise}
	$\triangleright$ Both $(\bZ/5\bZ)^*$ and $(\bZ/12\bZ)^*$ consist of $4$ elements. Write their multiplication tables, and prove that no re-ordering of the elements will make them match. (Cf. Exercise 1.6.) [$\S 4.3$]
\end{exercise}
\begin{solution}
	content...
\end{solution}