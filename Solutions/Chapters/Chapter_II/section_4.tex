\section{Group homomorphisms}
\extitle
\begin{exercise}
	$\triangleright$ Check that the function $\pi^n_m$ defined in $\S 4.1$ is well-defined and makes the diagram commute. Verify that it is a group homomorphism. Why is the hypothesis $m|n$ necessary? [$\S 4.1$]
\end{exercise}
\begin{solution}
	content...
\end{solution}

\begin{exercise}
	Show that the homomorphism $\pi_2^4 \times \pi_2^4\colon C_4 \to C_2 \times C_2$ is \emph{not} an isomorphism. In fact, is there \emph{any} isomorphism $C_4\to C_2\times C_2$?
\end{exercise}
\begin{solution}
	content...
\end{solution}

\begin{exercise}
	$\triangleright$ Prove that a group of order $n$ is isomorphic to $\bZ/n\bZ$ if an only if it contains an element of order $n$. [$\S 4.3$]
\end{exercise}
\begin{solution}
	content...
\end{solution}

\begin{exercise}
	Prove that no two of the groups $(\bZ, +), (\bQ, +), (\bR,+)$ are isomorphic to one another. Can you decide whether $(\bR,+),(\bC,+)$ \emph{are} isomorphic to one another? (Cf. Exercise $\RNo{6}.1.1$.)
\end{exercise}
\begin{solution}
	content...
\end{solution}

\begin{exercise}
	Prove that the groups $(\bR\smallsetminus\{0\},\cdot)$ and $(\bC\smallsetminus\{0\}, \cdot)$ are not isomorphic.
\end{exercise}
\begin{solution}
	content...
\end{solution}

\begin{exercise}
	We have seen that $(\bR,+)$ and $(\bR^{>0}, \cdot)$ are isomorphic (Example 4.4). Are the groups $(\bQ,+)$ and $(\bQ^{>0}, \cdot)$ isomorphic?
\end{exercise}
\begin{solution}
	content...
\end{solution}

\begin{exercise}
	Let $G$ be a group. Prove that the function $G\to G$ defined by $g\mapsto g^{-1}$ is a homomorphism if and only if $G$ is abelian. Prove that $g\mapsto g^2$ is a homomorphism if and only if $G$ is abelian.
\end{exercise}
\begin{solution}
	content...
\end{solution}

\begin{exercise}
	$\neg$ Let $G$ be a group, and let $g\in G$. Prove that the function $\gamma_g\colon G \to G$ defined by $(\forall a\in G): \gamma_g(a)=gag^{-1}$ is an automorphism of $G$. (The automorphisms $\gamma_g$ are called `inner' automorphisms of $G$.) Prove that the function $G \to \Aut(G)$ defined by $g\mapsto \gamma_g$ is a homomorphism. Prove that this homomorphism is trivial if and only if $G$ is abelian. [6.7, 7.11, $\RNo{4}.1.5$]
\end{exercise}
\begin{solution}
	content...
\end{solution}

\begin{exercise}
	$\triangleright$ Prove that if $m$, $n$ are positive integers such that $\gcd(m,n) = 1$, then $C_{mn} \cong C_m \times C_n$. [$\S 4.3$, 4.10, $\S\RNo{4}.6.1$, $\RNo{5}.6.8$]
\end{exercise}
\begin{solution}
	content...
\end{solution}

\begin{exercise}
	$\triangleright$ Let $p\neq q$ be odd prime integers; show that $(\bZ/pq\bZ)^{\ast}$ is not cyclic. (Hint: Use Exercise 4.9 to compute the order $N$ of $(\bZ/pq\bZ)^{\ast}$, and show that no element can have order $N$.) [$\S 4.3$]
\end{exercise}
\begin{solution}
	content...
\end{solution}

\begin{exercise}
	$\triangleright$ In due time we will prove the easy fact that if $p$ is a prime integer, then the equation $x^d = 1$ can have at most $d$ solutions in $\bZ/p\bZ$. Assume this fact, and prove that the multiplicative group $G = (\bZ/p\bZ)^*$ is cyclic. (Hint: Let $g\in G$ be an element of maximal order; use Exercise 1.15 to show that $h^{|g|} = 1$ for all $h\in G$. Therefore\ldots .) [$\S4.3$, $4.15$, $4.16$, $\S\RNo{4}.6.3$]
\end{exercise}
\begin{solution}
	content...
\end{solution}

\begin{exercise} % I cannot make the \neg be before the bullet point like in the book and it's driving me insane
	\begin{itemize}[wide]
		$\neg$
		\item Compute the order of $[9]_{31}$ in the group $(\bZ/31\bZ)^*$.
		\item Does the equation $x^3- 9 = 0$ have solutions in $(\bZ/31\bZ)^*$? (Hint: Plugging in all 31 elements of $(\bZ/31\bZ)^*$ is too laborious and will not teach you much. Instead, use the result of the first part: if $c$ is a solution of the equation, what can you say about $|c|$?) [$\RNo{7}.5.15$]
	\end{itemize}
\end{exercise}
\begin{solution}
	content...
\end{solution}
\begin{exercise}
	$\neg$ Prove that $\Aut_{\srf{Grp}}(\bZ/2\bZ\times \bZ/2\bZ) \cong S_3$. [$\RNo{4}.5.14$]
\end{exercise}
\begin{solution}
	content...
\end{solution}

\begin{exercise}
	$\triangleright$ Prove that the order of the group of automorphisms of a cyclic group $C_n$ is the number of positive integers $r\leq n$ that are relatively prime to $n$. (This is called \emph{Euler's $\phi$-function}; cf. Exercise 6.14.) [$\S\RNo{4}.1.4$, $\RNo{4}.1.22$, $\S\RNo{4}.2.5$]
\end{exercise}
\begin{solution}
	content...
\end{solution}

\begin{exercise}
	$\neg$ Compute the group of automorphisms of $(\bZ, +)$. Prove that if $p$ is prime, then $\Aut_{\srf{Grp}}(C_p)\cong C_{p-1}$. (Use Exercise 4.11.) [$\RNo{4}.5.12$]
\end{exercise}
\begin{solution}
	content...
\end{solution}

\begin{exercise}
	$\neg$ Prove \textit{Wilson's theorem: an integer $p>1$ is prime if and only if} 
	\[
		(p-1)! \equiv -1 \mod p.
	\]
	(For one direction, use Exercises 1.8 and 4.11. For the other, assume $d$ is a proper divisor of $p$, and note that $d$ divides $(p-1)!$; therefore\ldots .) [$\RNo{4}.4.11$]
\end{exercise}
\begin{solution}
	content...
\end{solution}

\begin{exercise}
	For a few small (but not too small) primes $p$, find a generator of $(\zmod{p})^{*}$.
\end{exercise}
\begin{solution}
	content...
\end{solution}

\begin{exercise}
	Prove the second part of Proposition 4.8.
\end{exercise}
\begin{solution}
	content...
\end{solution}