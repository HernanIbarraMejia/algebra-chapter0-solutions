

\documentclass{report}
%I'm using the package cm-super implicitly since I had some trouble displaying \S in the same font as in the book. Let me know if you're having the same issue. 
%Basic Maths
\usepackage{amsmath}
\usepackage{amssymb}
\usepackage{mathtools}
%For definining theorem-like environments
\usepackage{amsthm}
%For drawing commutative diagrams
\usepackage{tikz-cd}
%For beautiful letters (e.g. for a partition, see $\mathscr{P}$)
\usepackage{mathrsfs}

\usepackage{lipsum}


%Fix section numbering to match the book's convention
\renewcommand\thesection{\arabic{section}}

%Displays "Exercises". To put after each section.
\newcommand{\extitle}{\subsection*{Exercises}}

%Roman numerals!
\newcommand{\RN}[1]{%
	\textup{\uppercase\expandafter{\romannumeral#1}}%
}

%Exercise environment
\theoremstyle{definition}
\newtheorem{exercise}{}[chapter]

%Solution environment
\newenvironment{solution}
{\begin{proof}[Solution]}
	{\end{proof}}


\begin{document}
	\chapter{Preliminaries: Set theory and categories}
	\section{Naive set theory}
	\extitle
	\begin{exercise}
		Locate a discussion of Russell's paradox, and understand it.
	\end{exercise}
	\begin{solution}
	\end{solution}
	\begin{exercise}
		$\triangleright$ Prove that if $\sim$ is a relation on a set $S$, then the corresponding family $\mathscr{P}_{\sim}$ defined in $\S1.5$  is indeed a partition of $S$: that is, its elements are nonempty, disjoint, and their union is $S$. [$\S1.5$]
	\end{exercise}
	\begin{solution}
	\end{solution}
	\begin{exercise}
		$\triangleright$ Given a partition $\mathscr{P}$ on a set $S$, show how to define an equivalence relation $\sim$ on $S$ such that $\mathscr{P}$ is the corresponding partition. [$\S1.5$]
	\end{exercise}
	\begin{solution}
		content...
	\end{solution}
	\begin{exercise}
		How many different equivalence relations may be defined on the set $\{1,2,3\}$?
	\end{exercise}
	\begin{solution}
		content...
	\end{solution}
	\begin{exercise}
		Give an example of a relation that is reflexive and symmetric but not transitive. What happens if you attempt to use this relation to define a partition on the set? (Hint: Thinking about the second question will help you answer the first one.)
	\end{exercise}
	\begin{solution}
		content...
	\end{solution}
	\begin{exercise}
		$\triangleright$ Define a relation $\sim$ on the set $\mathbb{R}$ of real numbers by setting $a \sim b \iff b-a\in \mathbb{Z}$. Prove that this is an equivalence relation, and find a `compelling' description for $\mathbb{R}/{\sim}$. Do the same for the relation $\approx$ on the plane $\mathbb{R} \times \mathbb{R}$ defined by declaring $(a_1,a_2) \approx (b_1,b_2) \iff b_1 - a_1 \in \mathbb{Z}\textnormal{ and }b_2 - a_2\in\mathbb{Z}$. [$\S \RN{2}.8.1,\RN{2}.8.10$]
	\end{exercise}
	\begin{solution}
		content...
	\end{solution}
	\section{Functions between sets}
	\extitle
	\section{Categories}
	\extitle
	\section{Morphisms}
	\extitle
	\section{Universal properties}
	\extitle
	
\end{document}